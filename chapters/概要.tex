\chapter*{研究概要}
本論文では、観測データの解析で正しい値を得るために模擬データの解析を行うことで太陽系と太陽近傍星の運動を調べ、その結果を踏まえて観測データ解析も行いオールト定数と太陽運動を測定した。Oort-Lindbladモデルを用いることでオールト定数と太陽運動を得ることができるが、モデルでは速度分散が非常に小さい速度場を仮定している一方実際の銀河円盤星は無視できない程度の速度分散を持っていることが近年明らかになっている。また、解析の際には十分な星の数が必要だったり、モデルでは6次元位相空間 (位置の3次元と速度の3次元) データを用いるものの、観測データの不足から、そのうちの1次元 (視線速度) が含まれていない解析がなされていたりと、先行研究の解析にはまだ問題点がある。さらに、太陽運動の銀河回転方向成分$V_{\odot}$はまだ不定性が大きいことも問題である。

そこで、模擬データの生成・解析から速度分散の大きさ、視線速度の有無、サンプル数などの解析結果への効果を調べたところ、特に速度分散によるasymmetric driftの影響が大きく、古い星の解析で$V_{\odot}$を非常に大きく値を間違えて測定することが判明した。その他の影響として、サンプル数と視線速度の有無が解析の精度に対して大きく影響することが分かった。

観測データ解析では、Gaia Data Release 2から約100,000個の星の位置天文データから、局所静止基準に対する太陽系の運動とオールト定数の決定を行った。先行研究の太陽系運動とオールト定数の測定では、解析に視線速度を用いていなかったり、星の選び方にムラがあったりと先行研究で推定された星の年齢の値を用いることで、年齢ごとのサンプルで解析を行った結果、どのパラメータも年齢依存性を持つことがわかった。特に太陽運動の銀河回転方向成分$V_{\odot}$はサンプルの年齢と強い正の相関があり、これはにasymmetric driftの効果によるものであると考えられ、この傾向は模擬データで予想された結果と一致した。また、asymmetric driftでは天の川銀河のスケール長に依存することから、使用するスケール長の値によって得られる$V_{\odot}$の値は異なるものが得られた。さらに、年齢依存性はあるものの、先行研究と比べて妥当な値が得られた。最後に、本論文ではスケール長を年齢の関数で表して解析することで$V_{\odot}$の年齢依存性を解消することを目指した。