\chapter{まとめと議論 \label{chapSummary}}
本研究では、オールト定数と太陽運動をOort-Lindbladモデルを用いて解析する際のいくつかの効果を模擬データで調べ、同時に観測データで実際に解析を行った。

Oort-Lindbladモデルは天体の速度分散を無視できるような系を仮定しているモデルであり、また非軸対称ポテンシャルにも適用できるモデルである。しかし、本研究では軸対称ポテンシャルを仮定しているasymmetric driftをモデルに組み込むことで、そこの一貫性は失われていると言わざるを得ない。しかし、先行研究を見ても非対称性を表すオールト定数$C、K$の値は総じてその他のオールト定数$A、B$に比べて小さく、非軸対称性は決して大きくないことが分かっている。そのため、本研究で行っているモデルの改変は致命的に一貫性を欠いているとは言えないと考える。

模擬データの解析から、asymmetric driftの効果は太陽運動の銀河回転成分$V_{\odot}$の測定に非常に大きく影響することが分かった。また、サンプル数も解析結果に影響し、1000個以上のサンプル数であればオールト定数$A,B$と太陽運動の合計5パラメータについては10\%以上の精度を確保できることが分かった。

観測データの解析から、太陽運動$V_{\odot}$はスケール長$h_R,h_{\sigma}$の値に左右されることが分かった。これは、asymmetric driftの式の$1/h_R + 2/h_{\sigma}$の項が影響しているためであると思われる。図\ref{multi}の結果から、太陽運動$V_{\odot}$の値はasymmetric driftを考慮することによって考慮しないときよりも総じて小さくなることがわかった。観測データの解析5では6\,Gyrよりも若い星については年齢依存性を排除できたものの、6 Gyrよりも古い星では年齢依存性を持つ結果となった。解析4では年齢依存性を弱くすることができたが、多少の依存性は残っている。

\chapter{結論}
